\documentclass{article}
\usepackage{amsmath}
\usepackage{algorithm}
\usepackage{algorithmic}
\usepackage{listings}
\usepackage{xcolor}
\usepackage{amsfonts}
\usepackage{amssymb}
\usepackage{amsthm}
\usepackage{minted}

\definecolor{codegreen}{rgb}{0,0.6,0}
\definecolor{codegray}{rgb}{0.5,0.5,0.5}
\definecolor{codepurple}{rgb}{0.58,0,0.82}
\definecolor{backcolour}{rgb}{0.95,0.95,0.92}

\lstdefinestyle{mystyle}{
    backgroundcolor=\color{backcolour},   
    commentstyle=\color{codegreen},
    keywordstyle=\color{magenta},
    numberstyle=\tiny\color{codegray},
    stringstyle=\color{codepurple},
    basicstyle=\ttfamily\footnotesize,
    breakatwhitespace=false,         
    breaklines=true,                 
    captionpos=b,               
    morekeywords={def,while,len,(),Node,for,in,or,and,return},
    keepspaces=true,                 
    numbers=left,                    
    numbersep=5pt,                  
    showspaces=false,                
    showstringspaces=false,
    showtabs=false,                  
    tabsize=2
}

\lstset{style=mystyle}

\begin{document}

\title{Algorithm Design: 5th Hands On}
\author{M. Baglioni, L. Crociani, A. F. Montagnoli, G. Trapani}
\date{22 Apr 2024}
\maketitle

\section{Exercise 1}
\textit{Consider the counters $F[i]$ for $1 \leq i \leq n$, where $n$ is the number of items in the stream of any length. At any time, we know that $||F||$ is the total number of items (with repetitions) seen so far, where each $F[i]$ contains how many times item $i$ has been so far. We saw that CM-sketches provide a FPTAS $F'[i]$ such that $F[i] \leq F'[i] \leq F[i] + \epsilon ||F||$, where the latter inequality holds with probability at least $1 - \delta$.\\\\
Consider now a range query $(a,b)$, where we want $F_{ab} = \sum_{a \leq i \leq b} F[i]$. Show how to adapt CM-sketch so that a FPTAS $F'_{ab}$ is provided:
\begin{itemize}
    \item Baseline is $\sum_{a \leq i \leq b} F'[i]$, but this has drawbacks as both time and error grows with $b-a+ 1$.
    \item Consider how to maintain counters for just the sums when $b-a+ 1$ is any power of 2 (less or equal to $n$):
    \begin{itemize}
        \item[$\circ$] Can we now answer quickly also when $b-a+ 1$ is not a power of two?
        \item[$\circ$] Can we reduce the number of these power-of-2 intervals from $n log n$ to $2n$?
        \item[$\circ$] Can we bound the error with a certain probability? Suggestion: it does not suffices to say that it is at most $\delta$ the probability of error of each individual counter; while each counter is still the actual wanted value plus the residual as before, it is better to consider the sum $V$ of these wanted values and the sum $X$ of these residuals, and apply Markov's inequality to $V$ and $X$ rather than on the individual counters.
    \end{itemize}
\end{itemize}}
\subsection{Solution}
We shall assume w.l.o.g. $|F| = n$ is a power of 2 and build a vector of $\log_2 n$ CM-sketches.
Now we explain how to implement the Update and the RangeQuery operations.
\\\\
\textbf{Update}\\
We assume we have an operator \verb|sketch_update(cm, h, x)| which applies the Update operation
as defined in class on the CM-sketch \verb|cm| with the hash function \verb|h(x)|.

We can now give the pseudocode for the Update operation on our sketches, which takes as input
the CM-sketches \verb|cms|, the \(r\) hash functions \verb|hs| and the value \verb|x| we get from the stream.

\begin{lstlisting}
def Update(cms, hs, x):
    for i in [0; log_2(n) - 1]:
        for j in [0; r - 1]:
            sketch_update(
                cms[i], // We update the i-th sketch
                hs[j], // We apply the j-th hash function
                floor(x / pow(2, i))
            )
\end{lstlisting}
\  \\
\textbf{RangeQuery($a,b$):}\\
We can reduce every range-query to $2\log n$ point-queries over the CM-sketches defined as above.\\
Therefore we can partition RangeQuery($a,b$) in $2\log n$ point-queries and compute the sum of the results.\\
Complexity: $O(\log n\times r) = O\left(\log n\ \times \log\dfrac{1}{\delta}\right)$
%Let's make an example:\\\\
%\indent Considering $n=256$\\\\
%\indent $RangeQuery(42,99) = query(42) + query(43,64) + query(65,84) + query(85,96) + \indent\indent\indent\indent\indent\indent\indent\,\,\,\,\,\,\, query(97,98) + query(99)$
\subsubsection{Error Analysis}
We know with probability $\geq 1-\delta$ when we compute query($i$),
we get the following estimation
\begin{displaymath}
    \Tilde{F}[i]\leq F[i]+\epsilon ||F||
\end{displaymath}
We also know that
\begin{displaymath}
    \Tilde{F}[i] = F[i]+ X_{ji}
\end{displaymath} with $X_{ji}$ r.i.i.d which measures the error for the i-th element in row j.
\\\\
We know the expected value of $X_{ji}$ is
\begin{displaymath}
    E[X_{ji}] = \displaystyle\frac{\epsilon}{e}||F||
\end{displaymath}
\\
Therefore the expected value for the additive error of RangeQuery($a,b$)
which we compute calling the query method $2\log n$ times is
\begin{displaymath}
    2\log n\frac{\epsilon}{e}||F||
\end{displaymath}

Now, let $Y_{ji}$ be the error for the RangeQuery method.

Using Markov's inequality we can say that
\begin{displaymath}
    \Pr\left[Y_{ji} > 2\log n\epsilon ||F||\right]
    \leq\frac{E[Y_{ji}]}{2\log n\epsilon ||F||} =
    \frac{2\log n\dfrac{\epsilon}{e}||F||}{e \times 2\log n \times E[X_{ji}]} = \frac{2\log n E[X_{ji}]}{2e\log n E[X_{ji}]} = \dfrac{1}{e}
\end{displaymath}
Since in every CM-sketch we have $r = \ln\delta^{-1}$ rows:
\begin{displaymath}
    %\Pr	\left[\min_{j\in [r]}Y_{ji} > 2\log n\epsilon ||F||\right] =
	\prod_{j\in[r]}\Pr	\left[Y_{ji}>2\log n\epsilon ||F||\right] < \left(\dfrac{1}{e}\right)^r = \delta
\end{displaymath}
Therefore we demonstrated that:
\begin{displaymath}
    \text{RangeQuery}(F, a,b)\leq \text{RangeQuery}(\Tilde{F}, a,b)
\end{displaymath}
and that with probability $1-\delta$ we have:
\begin{displaymath}
    \text{RangeQuery}(\Tilde{F}, a,b)\leq \text{RangeQuery}(F, a,b) + \epsilon \times 2\log n||F||  
\end{displaymath} \qed


We observe that in order to estimate with correctness up to $\epsilon'||F||$
with probability $1-\delta$ it is sufficient to choose a value $\epsilon = \displaystyle\frac{\epsilon'}{2\log n}$.



\section{Exercise 2 (Bonus)}

\textit{Show (and prove correctness) that there is a deterministic streaming algorithm that works in O(1) space and finds the most frequent item if the latter appears strictly more than half of the times in the stream.}

\subsection{Solution}

The address the problem we have to iterate O(n) times on the stream. Follows a pseudo-coded algorithm:

\begin{lstlisting}
def find_majority(stream):
    maj = None
    counter = 0
    while stream.has_next():
        elem = stream.next()
        if counter == 0 or maj is None or elem == maj:
            maj = elem
            counter += 1
        else:
            counter -= 1
    return maj
\end{lstlisting}

\newpage

\subsubsection{Correctness}

%To prove the correctness of the algorithm, we shall use an invariant argument.\\\hfill
%
%\noindent \textbf{Invariant}: During the execution of the algorithm, the current element $maj$ is the majority element of the stream up to that point, if there exists a majority element.\\
%
%\noindent \textbf{Base of Induction}: At the beginning of the algorithm, $maj$ is initialized to $None$, so we cannot say if there is a majority element. However, since the invariant only requires that $maj$ be the majority element up to that point if one exists, the base of induction is satisfied.\\\hfill
%
%\noindent \textbf{Inductive Step}: Suppose the invariant is true before an iteration of the loop. There are two cases to consider:
%
%\begin{enumerate}
%    \item If $counter$ is zero or if $maj$ is equal to the next element in the stream, then the current element $maj$ is the majority element up to that point. After updating $maj$ and incrementing $counter$, the invariant still holds for the next iteration of the loop.
%    \item If $counter$ is not zero and $maj$ is not equal to the next element in the stream, it means that $maj$ is not the majority element up to that point. However, since each pair of different elements is "eliminated" by decrementing $counter$, the current element $maj$ continues to be the majority element relative to the remaining elements in the stream. Thus, the invariant still holds for the next iteration of the loop.
%\end{enumerate}
%
%\noindent Continuing this process, we can conclude that the invariant holds for all iterations of the loop. Thus, the algorithm is correct and will correctly return the majority element if it appears more than half of the times in the stream.


We shall prove the correctness of this algorithm via induction.

\textbf{Inductive hypothesis}: the algorithm returns the majority element up to
length \(< l\) if it exists.

\textbf{Base Case}: the length of the stream is 1, there is only one element
in the stream, and it is returned correctly.

\textbf{Inductive step}: We distinguish two cases:
\begin{itemize}
	\item In this case the element we are considering is not the majority element.
	The counter becomes 0 at some point. As it becomes 0, the algorithm
	``resets'' and starts from scratch on a length \(< l\).
	\item In this case the element we are considering was the majority element.
	We can have three different subcases:
	\begin{itemize}
		\item The counter increases. The majority element is the element we are
		considering now, and it is returned.
		\item The counter decreases but it is still greater than 0. The majority
		element is the element we are considering now, and it is returned.
		\item The counter decreases but it becomes 0.
		This is the first case we discussed: the element we are considering now
		is not the majority element.
	\end{itemize}
\end{itemize}


\end{document}
