\documentclass{article}
\usepackage{graphicx} % Required for inserting images
\usepackage{amsmath}
\usepackage{amsfonts}
\usepackage{amssymb}
\usepackage{amsthm}
\usepackage{minted}
\usepackage[table]{xcolor}
\title{Hands-On 8: Algorithm Design A.Y. 2024}

\author{Emanuele Buonaccorsi \and Suranjan Ghosh \and Giacomo Trapani}

\newtheorem*{remark}{Remark}
\newtheorem*{example}{Example}
\newtheorem*{theorem}{Theorem}
\newcommand{\argmax}{\mathop{\mathrm{argmax}}\limits}

\date{20 May 2024}
\begin{document}

\maketitle

\section{Exercise 1}
{\itshape
Two players drive up to the same intersection at the same time. If both attempt to cross, the
result is a fatal traffic accident. The game can be modeled by a payoff matrix where crossing
successfully has a payoff of 1, not crossing pays 0, while an accident costs -100.
\begin{itemize}
	\item Build the payoff matrix
	\item Find the Nash equilibria  
	\item Find a mixed strategy Nash equilibrium  
	\item Compute the expected payoff for one player (the game is symmetric)
\end{itemize}
}
\subsection{Solution}
\subsubsection*{Build the payoff matrix}
\begin{center}
	\begin{tabular}{|c|c|c|}
        \hline
        & Cross & Do not cross \\\hline
        Cross &  -100, -100 & 1, 0 \\\hline
        Do not cross & 0, 1 & 0, 0 \\\hline
  \end{tabular}
\end{center}

\subsubsection*{Find the Nash equilibria}
Nash equilibria are \textbf{Cross, Do not cross} ((1, 0), top right) and \textbf{Do not cross,
Cross} ((0, 1), bottom left).

\begin{center}
	\begin{tabular}{|c|c|c|}
        \hline
        & Cross & Do not cross \\\hline
        Cross &  -100, -100 & \cellcolor{green!30}1, 0 \\\hline
        Do not cross & \cellcolor{green!30}0, 1 & 0, 0 \\\hline
  \end{tabular}
\end{center}

\subsubsection*{Find a mixed strategy Nash equilibrium}
Call \(u_1\) the utility function for Player 1. To find a mixed strategy Nash equilibrium
the utility cost for action \textbf{Cross} and \textbf{Do not cross} must be the same.
\[
	u_1(\text{Cross}) = u_1(\text{Do not cross})
\]
In other words, we need to choose a \(p\) such that
\[
	\begin{aligned}
		& -100p + 1(1-p) = 0(p) + 0(1-p) \\
		& -100p + 1 - p = 0 \\
		& p = \dfrac{1}{101}
	\end{aligned}
\]
This means Player 1 is willing to randomize whether to cross or not if Player 2 is not crossing
\(100/101\) times and crossing otherwise.

Since the game is symmetric, the same reasoning can be applied for Player 2.
\qed

\subsubsection*{Compute the expected payoff for one player}
First we calculate the probability each outcome occurs.
\begin{center}
	\begin{tabular}{|c|c|c|}
        \hline
        & Cross & Do not cross \\\hline
        Cross &  \(1/10201\) & \(100/10201\) \\\hline
        Do not cross & \(100/10201\) & \(10000/10201\) \\\hline
  \end{tabular}
\end{center}

We choose a player and we multiply the probability of each outcome by its payoff and we
sum the results.
\[
	(-100) \times 1/10201 + 1 \times 100/10201 = 0.
\]

\section{Exercise 2}
{\itshape
Kicking a penalty, if the kicker sends the ball one side and the goalkeeper goes on the
same side, the latter will always be able to catch the ball and avoid the goal.
A certain kicker never fails when kicks left, instead he sometimes fails (say once
every four penalties) when kicks right. Should him prefer kicking left? Model this
situation as a game, describe a strategy both for the kicker and the goalkeeper and
compute the expected payoff for both players.
}

\subsection{Solution}
To reason about this problem, we must first build the payoff matrix. We also calculate the
expected payoff for both actions. We say Player 1 is the goalkeeper, Player 2 is the kicker,
and in order to be as generic as possible we say the kicker fails once every
\(x \in \mathbb{N^+}\) penalties.

We assume that "failing" means that the kicker instead of kicking right, actually kicks left.

\

\begin{table}[h!]
    \centering
    \renewcommand{\arraystretch}{1.5} % Increases the height of the rows
    \setlength{\tabcolsep}{15pt}      % Increases the width of the columns
    {\Large % Increases the font size within the group
    \begin{tabular}{|c|c|c|}
        \hline
        & Kicks L & Kicks R \\\hline
        Dives L &  1, 0 & \( \frac{1}{x}, \frac{x-1}{x} \) \\\hline
        Dives R & 0, 1 & 1, 0 \\\hline
    \end{tabular}
    }
    \caption{Payoff matrix, where player 1 (on the left) is the goalkeeper and player 2 (on top) is the kicker.}
    \label{tab:penalty-kick-game}
\end{table}

\

Let's call \(p\) the probability that the goalkeeper chooses to dive left. And let's call \(q\) the probability that the kicker chooses to kick left.
The expected payoff for the goalkeeper is (for both his/her possible actions):
\[
E_1(\text{Dives L}) = 1q + \frac{1}{x}(1-q)
\]
\[
E_1(\text{Dives R}) = 0q + 1(1-q) = 1 - q
\]

In order to be able to randomize its choices, the utility of both actions available must be
the same.
\[
	\begin{aligned}
        & E_1(\text{Dives L}) = E_1(\text{Dives R}) \\
        & \dfrac{q(x-1)+1}{x} = 1 - q \\
        & \dfrac{q(2x-1)}{x} = \dfrac{x-1}{x} \\
        & q = \dfrac{x-1}{2x-1}
	\end{aligned}
\]

\

Doing the same thing for the kicker:
\[
E_2(\text{Kicks L}) = 0p + 1p
\]
\[
E_2(\text{Kicks R}) = (1-p)\dfrac{x-1}{x} + 0p
\]

In order to make the utility of both actions the same:

\[ E_2(\text{Kicks L}) = E_2(\text{Kicks R}) \iff  p = \dfrac{x-1}{2x-1} \]

\

So, when in a mixed strategy Nash equilibrium, the kicker would actually prefer to kick right. For example, if x=4 (the kicker fails one out of 4 right kicks), the probability of kicking left would be \(q = 3/7 \approx 0.43\).

\section{Exercise 3}
{\itshape
Two cars stand on the opposite sides of a long bridge (as long as they cannot see the other
side) that is narrow enough that only one car can cross it. If both attempt to cross the bridge,
they will meet in the middle and remain stuck for several hours until the tow truck rescues them.
As an alternative, each driver may decide to follow a secondary road, which is much wider but
also much longer.  Model this situation as a game, find the Nash equilibrium and discuss a
possible mixed strategy.
}

\subsection{Solution}

\subsubsection*{Build the payoff matrix and find the equilibria}

We define these outcomes:

\begin{itemize}
    \item \(S\) (stuck) if the player is stuck
    \item \(F\) (fast) if the player is able to cross the fast road
    \item \(L\) (long) if the player crosses the long road
\end{itemize}

We then define the payoff matrix:

\begin{center}
    \begin{tabular}{|c|c|c|}
        \hline
                        & Cross & Do not cross \\\hline
        Cross           &  S, S & F, L \\\hline
        Do not cross    &  L, F & L, L \\\hline
    \end{tabular}
\end{center}

If we rank these outcomes by defining this ordinal payoff:

\[ F > L > S \]

The matrix becomes:

\begin{center}
    \begin{tabular}{|c|c|c|}
        \hline
                        & Cross & Do not cross \\\hline
        Cross           & (0, 0) & \cellcolor{green!30}(2, 1) \\\hline
        Do not cross    & \cellcolor{green!30}(1, 2) & (1, 1) \\\hline
    \end{tabular}
\end{center}

Where the ones in green are the Nash equilibria.


\subsubsection*{Mixed strategy}

\begin{itemize}
    \item Let \( p \) be the probability that Player 1 chooses "Cross."
    \item Let \( q \) be the probability that Player 2 chooses "Cross."
\end{itemize}

\

The expected payoffs for player one are:
\[
E_1(\text{Cross}) = q \cdot 0 + (1 - q) \cdot 2 = 2(1 - q)
\]
\[
E_1(\text{Do not cross}) = q \cdot 1 + (1 - q) \cdot 1 = 1
\]

The expected payoffs for player two are:
\[
E_2(\text{Cross}) = p \cdot 0 + (1 - p) \cdot 2 = 2(1 - p)
\]
\[
E_2(\text{Do not cross}) = p \cdot 1 + (1 - p) \cdot 1 = 1
\]

\

\noindent For a mixed strategy to be a Nash equilibrium, the players must be indifferent to the strategies they choose. This means the expected payoffs for both strategies should be equal.

For Player 1 to be indifferent:
\[
2(1 - q) = 1 \iff q = \frac{1}{2}
\]

For Player 2 to be indifferent:
\[
2(1 - p) = 1 \iff p = \frac{1}{2}
\]









\section{Exercise 4}
{\itshape
The Municipality of your city wants to implement an algorithm for the assignment of children
to kindergartens that, on the one hand, takes into account the desiderata of families and, on
the other hand, reduces city traffic caused by taking children to school. Every school has a
maximum capacity limit that cannot be exceeded under any circumstances. As a form of welfare,
the Municipality has established the following two rules:   
\begin{itemize}
	\item  in case of a child already attending a certain school, the sibling is granted the
	same school;
	\item families with only one parent have priority for schools close to the workplace.
\end{itemize}
Model the situation as a stable matching problem and describe the payoff functions of
the players.

Question: what happens to twin siblings?   
}

\subsection{Solution}
The idea is to re-use the \textit{stable marriage algorithm} presented during lecture. Our aim is to model the problem in a way that we can directly apply the algorithm to solve it, without changing anything in the original algorithm provided.

\

In the algorithm men and women play a different part, so our purpose shall be to model
the problem and define who are the men, who are the women and what criteria are they using to build their preferences lists. The solution we propose is to:
\begin{itemize}
	\item \textbf{define as a \textit{man} each of the kindergartens' available seats}, their preference
	list is ordered by the ``sibling condition'' i.e. each kindergarten \(K\) prefers a sibling
	of a child enrolled the previous year(s) over a child who has no siblings in \(K\);
	\item \textbf{define as a \textit{woman} the kids}, their preference list is ordered by the
	distance to each kindergarten.
\end{itemize}

Choosing such a model, we always satisfy the condition given in the first rule and we
give priorities according to the second rule.

It is evident this model cannot handle correctly twin siblings: we shall prove this
with a minimal example.

\begin{example}
	Assume we have 2 kindergartens and the both of them declare their available spots
	for the new year are 1. Suppose we have only 1 family with a single pair of twin
	siblings. Given the rules mentioned in the problem statement, we cannot allocate
	seats for the twin siblings because it is not possible for the both of them to
	attend the same kindergarten. \qed
\end{example}
\end{document}