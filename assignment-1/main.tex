\documentclass[a4paper, 12pt]{article}
\usepackage[top=2.5cm, bottom=2.5cm, left=3cm, width=15.09cm]{geometry}

\usepackage{mlmodern}
\usepackage[T1]{fontenc}
\usepackage[utf8]{inputenc}

\usepackage{mathtools,stmaryrd,galois,amsmath,amssymb,amsthm}
\usepackage{graphicx}
\usepackage{mleftright,xspace}
\usepackage{algorithm}
\usepackage{algorithmic}

\newtheorem{theorem}{Theorem}
\newtheorem{lemma}{Lemma}
\DeclarePairedDelimiter\floor{\lfloor}{\rfloor}

\title{Hands-On 1: Algorithm Design 23/24}
\date{17 March 2024}
\author{Filippo Boni \and Alessio Duè \and Giacomo Trapani}

\begin{document}

\maketitle

\section{Exercise 1}
{
\itshape{Describe and analyze how to transform the randomized Karp-Rabin string search algorithm from Monte Carlo to Las Vegas}.
}
\section*{Solution}

The main idea here is to consider the rolling hash technique, which in the standard implementation is a Monte Carlo algorithm (which may return false positives). Assuming that:

\[P \text{ is a pattern } P:[0, \dots, m-1]\]
\[T \text{ is a text } T:[0, \dots, n-1]\]

\subsection{Algorithmic Implementation}
A possible solution for the problem is to analyze algorithmically a version of the Karp-Rabin method that satisfies the given condition of no errors; here is the proposed approach:

\begin{lemma}
  Every positive integer \( a \) has at most \( \log_2 x \) distinct prime divisors.
\end{lemma}

\begin{proof}
  By the \textit{fundamental theorem of arithmetic}, we can write \( x \) as a product of primes \( X = \prod_{i=1}^{k} p_i \); since \( p_1 = 2 \), then \( x = \prod_{i=1}^{k} p_i \geq 2^k \). Therefore, it follows that \( k \leq \log_2 x \).
\end{proof}

\begin{lemma}
  Let \( \pi(x) \) be the cardinality of the set of prime numbers in the interval \([2;x]\). The value of \( \pi(x) \) for \( x \geq 2 \) is approximately \( \frac{x}{\log(x)} \).
\end{lemma}

\noindent
We use an operator \(\mathrm{hash}(s)\) that computes the hash for a string, and \(U(h, x)\) that updates the rolling hash \(h\) with a new character \(x\).

Here follows the pseudo-code for the Karp-Rabin Las-Vegas:

\begin{algorithm}[H]
\caption{KARP-RABIN(T, P, hash, p)}
\label{alg:karp-robin}
\begin{algorithmic}[1]
\STATE $m \leftarrow |P|$
\STATE $n \leftarrow |T|$
\STATE $target \leftarrow \mathrm{hash}(P)$
\STATE $current\_hash \leftarrow \text{None}$

\FOR{$i \in 1, \dots, n-m$}
\IF{$current\_hash \neq \text{None}$}
\STATE $current\_hash \leftarrow U( current\_hash, T[i+m])$
\ELSE
\STATE $current\_hash \leftarrow \mathrm{hash}(T[i:m])$
\ENDIF

\IF{$current\_hash = target$}
\IF{$\text{Check\_String\_Equality}(T[i:i+m], P)$}
\STATE Print("Match found")
\STATE \textbf{return} $(i, i+m)$
\ENDIF
\ENDIF
\ENDFOR

\STATE \textbf{return} None
\end{algorithmic}
\end{algorithm}

\subsection{Algorithm Analysis}
\subsubsection{False Matches}
\begin{theorem}
Assuming \( m \geq 1 \), \( n \geq 2 \), \( K \) constant, \( B \) a suitably large number in \( N \), \( p \) a randomly chosen prime number in the interval \([2, B]\), Karp-Rabin reports a false match for a given position with a probability of at most $\frac{1}{K}$
\end{theorem}

\begin{proof}
Let:
\[ \text{target} = \text{hash}(P) \]
\[ T_i = \text{hash}(T[i:i+m]); \]
Define:
\begin{itemize}
\item NoMatch to be the set of indices \( i \) for which \( T_i \neq \text{target} \), i.e., no match with target;
\item \( x = \prod_{i \in \text{NoMatch}} (|T_i - \text{target}|) \).
\end{itemize}

We know that \( x\) is a positive integer and since \( | \text{NoMatch} | \leq n-m \) and \( | T_i - \text{Target} | \leq 2^m \), we can say that \( x \leq 2^{m (n-m)} \leq 2^{mn} \) . By Lemma 1, we deduce that \( x \) has at most \( m \cdot n \) prime divisors. Since we have a false match when \( T_{i} \bmod p  = \text{target} \bmod p \), then if there were a false match, \( p \) would divide \( x \).

The following probability results are derived:
\[ \text{Pr}(\text{false match}) \leq \text{Pr}(p \text{ divides } x) = \frac{\text{number of prime divisors of } x}{\pi (B)} \leq \frac{mn}{\pi (B)} \]
Choosing \( B = K \cdot m \cdot n \), it follows that \( \text{Pr}(\text{false match}) = O\left(\frac{1}{K}\right) \).
\end{proof}

\subsubsection{Complexity Analysis}
\begin{description}
\item[Worst case] we have a false match at every position: checking \( m \) characters \( n \) times \(\Rightarrow O(m \cdot n)\).
\item[Best case] we have a true match in the first position: checking \( m \) characters once \(\Rightarrow O(m)\).
\item[Expected] the probability of a false match at each position is $O(1/K)$, so the expected number of false matches is $O(n / K)$. The expected running time is $O(nm/K + n + m)$. We can choose $K$ such that $m/K=c$ with $c$ constant; for instance \(K = m\). We then obtain $O(m + n)$.

  The running time is not influenced by the value of $K$, except for the generation of the prime $p$, which is an additive logarithmic term in the asymptotic complexity and can thus be ignored.
\end{description}

\section{Exercise 2}
{
	\itshape{Show that \(h_{ab}(x) = (a x + b \mod p) \mod m\) is 2-independent.}
}

\section*{Solution}
We call \(x_1\), \(x_2\) two values in \(U\) such that \(x_1 \neq x_2\).
We have to prove that
\[
	\Pr_{h \in \mathit{H}}[h(x_1) = y_1 \wedge h(x_2) = y_2] \leq \dfrac{1}{m^2}.
\]


First, we consider the linear system
\[
	\begin{aligned}
		\left\{ 
		\begin{array}{l}
			z_1 = ax_1 + b \mod p \\
			z_2 = ax_2 + b \mod p
		\end{array} 
	\right.
	\end{aligned}
\]

Since \(p\) is prime, we know from Lagrange's theorem there is only one mapping from pairs \((x_1, x_2)\) to
\((z_1, z_2)\), and we have \(p\times(p-1)\) possible pairs of \((z_1, z_2)\).

We notice the number of values in the same residual class modulo \(m\) has a lower bound in \(\floor{p/m}\)
and an upper bound in \(\floor{p/m} + 1\).

We can now give the result
\[
	\dfrac{1}{p(p-1)} \times \left(\floor{\dfrac{p}{m}}\right)^2 \leq \Pr_{h \in \mathit{H}}[h(x_1) = y_1 \wedge h(x_2) = y_2] \leq \dfrac{1}{p(p-1)} \times \left(\floor{\dfrac{p}{m}} + 1\right)^2
\]

We can approximate \(p(p-1)\) with \(p^2\) and get the upper bound we have in the thesis.\qed

\section{Exercise 3}
{
\itshape{Consider the deletion in cuckoo hashing: build an example so that the deletion does not produce a feasible graph, meaning that there is no insertion-only sequence that can lead to that graph. Show how to fix this by randomly choosing among \(h_1\) and \(h_2\) when inserting.}
}
\section*{Solution}

Let's consider a minimal example that shows how deletions can produce a graph which is not obtainable by insertions alone. Let \(\mathcal H = \left\{ h_{a,b} \mid a \in \mathbb Z_{p}^+, b \in \mathbb Z_{m} \right\}\), \(h_{a,b}(x) = ((ax + b) \bmod p) \bmod m\), \(p=3\), \(m=2\), \(h_1 = h_{1,0} \in \mathcal H\), \(h_2 = h_{1,1} \in \mathcal H\). The hash table has 2 slots and contains value in \(\mathbb Z_{3}\). Their hashes are:
\[\begin{array}{r|c|c|c}
  x & 0 & 1 & 2 \\\hline
  h_1 & 0 & 1 & 0 \\\hline
  h_2 & 1 & 0 & 0 \\
\end{array}\]

The sequence of operations \texttt{insert 2, insert 0, delete 2} yields the table \texttt{[empty, 0]}. This is not obtainable via insertions alone, because if the first position is empty, \texttt{0} will be inserted there.

We can reach such a configuration if we randomize the choices between the hash functions. For example,
the configuration written above can be reached via insertions with \texttt{insert 0} if the hash function
\(h_2\) can be chosen.

Consider the configuration in the picture below.

\includegraphics*[scale=0.24]{./images/cuckoo-hashing.png}

Assume red is inserted before blue. We have a conflict for position \(i\) and if we deterministically
choose the hash function and make a deletion we can get a table which is not achievable via insertions.

However, if we randomize the choice, we can prove there exists a sequence of random insertions s.t. the table
we get at the end of the insertions is identical to a certain previously unobtainable configuration.
\end{document}
